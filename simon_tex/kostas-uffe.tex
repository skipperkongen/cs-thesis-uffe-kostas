%-shell-escape to create figures
%\write18{rm -f cap.eps ; fig2dev -L eps cap.fig cap.eps}

\documentclass[a4paper,11pt,final]{article}
%%%%%%%%%%%%%%%%%%%%%%%%%%%%%%%%%%%
% Alle pakkerne
%%%%%%%%%%%%%%%%%%%%%%%%%%%%%%%%%%

\usepackage{a4wide}
%\usepackage{pstricks}
\usepackage{amsfonts}
\usepackage[latin1]{inputenc}
%\usepackage{mathptm}
%\usepackage{fixme}
\usepackage{amsmath}
\usepackage{amssymb}
\usepackage{amsthm}
\usepackage{dsfont}
%\usepackage{clrscode}
\usepackage[final]{graphicx}
%\usepackage{boxedminipage}
%\usepackage{longtable}
%\usepackage{varioref}
%\vrefshowerrors
%\usepackage{placeins}
%\usepackage{rotating}
\usepackage{fancyhdr}
\usepackage{lastpage}
%\usepackage{booktabs}
\usepackage[all]{xy}
\usepackage{float}
\usepackage{lscape}
%\usepackage[nottoc]{tocbibind}
\usepackage{url}
\usepackage{eqnarray}
\usepackage{subeqnarray}
\usepackage{times}
\usepackage{latexsym}
\usepackage[english]{babel}
%\usepackage[numbers]{natbib}


%%%%%%%%%%%%%%%%%%%%%%%%%%%%%%%%%%%%%
% Definitioner, teoremer ovs.
%%%%%%%%%%%%%%%%%%%%%%%%%%%%%%%%%%%%%
% \newtheorem{theo}{Theorem}
% \newtheorem{defi}[theo]{Definition}
% \newtheorem{prop}[theo]{Proposition}
% \newtheorem{procedure}[theo]{Procedure}
% \newtheorem{obs}[theo]{Observation}
% \newtheorem{con}[theo]{Conjecture}

%%%%%%%%%%%%%%%%%%%%%%%%%%%%%%%%%%%%%
% Figure og tabel numre
%%%%%%%%%%%%%%%%%%%%%%%%%%%%%%%%%%%%%
%\numberwithin{equation}{section}
%\numberwithin{figure}{section}
%\numberwithin{table}{section}

\newcommand{\slsp}{MLMR-CLSSP}
\newcommand{\sit}{MIMR-CLSSP}
\newcommand{\blsp}{MLMR-CLSP}
\newcommand{\bres}{ML-CLSP}
\newcommand{\bit}{MIMR-CLSP}
\newcommand{\bitres}{MI-CLSP}

\newcommand{\NP}       {\mbox{$\cal N\rule{-1mm}{0mm}P$}}
\newcommand{\note}[1]{\rule{0mm}{0mm}\\ ({\red SS: #1})}

%%%%%%%%%%%%%%%%%%%%%%%%%%%%%%%%%%%%%%
% Header og footer
%%%%%%%%%%%%%%%%%%%%%%%%%%%%%%%%%%%%%%
\usepackage{fancyhdr}
\addtolength{\headheight}{2pt} \pagestyle{fancy}
\renewcommand{\headrulewidth}{0.4pt}
\renewcommand{\footrulewidth}{0pt}
\rhead{Spoorendonk}
%\fancyhead[LE]{\nouppercase{\leftmark}}
%\fancyhead[RO]{\nouppercase{\rightmark}}
%\fancyfoot[RO,LE]{\thepage}
\lhead{Multi-Level, Multi-Resource Lot Sizing Problems}


 %%%%%%%%%%%%%%%%%%%%%%%%%%%%%%%%%%%%%%
% Example enviroment
%%%%%%%%%%%%%%%%%%%%%%%%%%%%%%%%%%%%%%
\floatstyle{ruled}
%\newcounter{ex}
%\setcounter{ex}{1}
\newfloat{ex}{tbhp}{aux}
\floatname{ex}{Example}

%% to do cite with names and number, Author [?]
\renewcommand{\cite}[1]{\citet{#1}}

%%%%%%%%%%%%%%%%%%%%%%%%%%%%%%%%%%%%%
% Titel og alt det stas
%%%%%%%%%%%%%%%%%%%%%%%%%%%%%%%%%%%%%
\title{Multi-Level, Multi-Resource Lot Sizing Problems}
\author{
Simon Spoorendonk \\
\texttt{spooren@diku.dk}\\
\emph{DIKU Department of Computer Science, University of Copenhagen}\\
\emph{Universitetsparken 1, DK-2100 Copenhagen �, Denmark} \\
}

\begin{document}
\maketitle

\begin{abstract}
Presentation of models and basic heuristics for Multi-Level, Multi-Resource Lot Sizing Problems.

%\paragraph{Keywords:} Lot Sizing Problems
\end{abstract}


\section{Modelling}

\subsection{Sets}
\begin{itemize}
\item $I$ is the set of items
\item $S(i)$ is the set of items that uses $i$ in their production, i.e., one
  level below in the bill of material (BOM)
\item $K$ is the set of resources
\item $T$ is the set of time periods
\end{itemize}

\subsection{Variables}
\begin{itemize}
\item $s_t^i$ is the number of units of item $i$ on stock at time $t$
\item $x_t^{ik}$ is the number of units of production of item $i$ on resource $k$ at
  time $t$. No resource ($k$) index for big-bucket model.
\item $y_t^{ik}$ indicates if resource $k$ is set up for production of
  item $i$ at time $t$. No resource ($k$) index for big-bucket model.
\item $z^{ik}_t$ indicates if resource $k$ is being set up to produce item $i$ at time $t$, and is thus consuming capacity at time $t$
\end{itemize}

\subsection{Constants}
\begin{itemize}
\item $h^i_t$ is the unit inventory cost of item $i$ at time $t$
\item $p^{ik}_t$ is the unit production cost of item $i$ on resource $k$ at time $t$
\item $f_t^{ik}$ is the fixed cost of resource $k$ for having it set up
  for item $i$ at time $t$
\item $d_t^i$ is the demand of item $i$ at time $t$
\item $r^{ij}$ is the number of units of item $j$ used for the production of
  item $i$
\item $\alpha^{ik}_t$ is the unit capacity for producing item $i$ at
  resource $k$ at time $t$
\item $\beta^{ik}_t$ is the capacity used for setting up the
  production of item $i$ at
  resource $k$ at time $t$
\item $C^k_t$ is the capacity of resource $k$ at time $t$
\item $M$ is a large constant
\item $\gamma^{ik}_t$ is the amount of time it takes to set produce item $i$ on  resource $k$ at time $t$.
\end{itemize}

\subsection{Small-bucket}
Multi-Level Multi-Resource Capacitated Lot Sizing and Scheduling Problem with
Setup Times (\slsp):
  \begin{align}
\label{eq:s:obj}
\min ~ & \sum_{i\in I} \left [ h^i_0 s_0^i + \sum_{t\in
T} \left ( h^i_t s_t^i + \sum_{k\in K} ( p^{ik}_t x_t^{ik} +
          f^{ik}_t y_t^{ik}) \right ) \right ] \\
\label{eq:s:flow} \text{s.t. } & s_{t-1}^i + \sum_{k\in
K}x_{t-\gamma^{ik}}^{ik} = d_t^i + \sum_{j \in
S(i)} \sum_{k\in K} r^{ij} x_t^{jk}  +
    s_t^i &  t \in T, ~ i \in I \\
\label{eq:s:cap} &   \alpha^{ik}_t x_t^{ik} + \beta^{ik}_t z_t^{ik}
\leq C_t^{k} y_t^{ik} & t \in T, ~ i \in I, ~  k \in K \\
\label{eq:s:one} &    \sum_{i \in I} y_t^{ik} \leq 1 &  t \in T
~ , k \in K    \\
\label{eq:s:setup} &    y_t^{ik} \geq z_t^{ik} \geq y_t^{ik}- y_{t-1}^{ik} &
t \in T, ~ i \in I, ~ k \in K\\
\label{eq:s:bound} &    s_t^i, s_0^i \geq 0,~x_t^{ik} \in \mathds{Z}^+ ,~ y_t^{ik}, z_t^{ik} \in \mathds{B} &t \in T ,~ i \in I,~
k \in K
  \end{align}
A big issue when solving this model is the possibility for an item to
be produced in parallel on different resources.

If $r^{ij}=0$ for all items $i,j$ then \eqref{eq:s:flow} can be
rewritten to:
  \begin{align}
\label{eq:s:flow1} \text{s.t. } & s_{t-1}^i + \sum_{k\in
K}x_{t-\gamma^{ik}}^{ik} = d_t^i +  s_t^i &  t \in T, ~ i \in I
  \end{align}
That is, by substituting \eqref{eq:s:flow} with \eqref{eq:s:flow1} one
obtains the Multi-Item Multi-Resource Capacitated Lot Sizing and Scheduling Problem with
Setup Times (\sit).

\subsection{Big-Bucket}
By aggregating the time slots in the small-bucket model one obtains a
big-bucket-model. Now items can share resources in the same time slot
and an item can consume capacity of several resources in the same time
slot (this is not equivalent with parallel production as described
for the small-bucket model, but is an estimate of the production of a
big-bucket item consisting of part of a product structure for a
small-bucket item). Multi-Level Multi-Resource Capacitated Lot Sizing Problem with
Setup Times (\blsp):
\begin{align}
\label{eq:big:obj}
\min ~ & \sum_{i\in I} \left ( h^i_0 s_0^i + \sum_{t\in
T} ( h^i_t s_t^i + p^i_t x_t^i +
          f^i_t y_t^i ) \right ) \\
\label{eq:big:flow} \text{s.t. } & s_{t-1}^i + x_{t-\gamma^i}^i = d_t^i + \sum_{j \in
S(i)} r^{ij} x_t^j  +
    s_t^i & t \in T, ~ i \in I \\
\label{eq:big:prod}
& x_t^i \leq M y_t^i & t \in T, ~  i \in I \\
\label{eq:big:cap} & \sum_{i \in I} \left (  \alpha^{ik}_t x_t^i +
  \beta^{ik}_t y_t^i \right )
\leq C_t^{k}  & t \in T, ~  k \in K \\
\label{eq:big:bound} &    s_t^i, s_t^0,x_t^i \geq 0,~ y_t^i \in \mathds{B} &t \in T ,~i \in I
\end{align}

This can be aggregated further into a single resource.
\begin{align}
\label{eq:big:cap1} & \sum_{i \in I} \left (  \alpha^i_t x_t^i +
  \beta^i_t y_t^i \right )
\leq C_t  & t \in T
\end{align}
Substitute \eqref{eq:big:cap} with \eqref{eq:big:cap1} to obtain the
Multi-Level Capacitated Lot Sizing Problem with Setup Times (\bitres).

By aggregating time further one can aim for a single-level product
structure.
\begin{align}
\label{eq:big:flow1} \text{s.t. } & s_{t-1}^i + x_{t-\gamma^i}^i = d_t^i + s_t^i & t \in T, ~ i \in I
\end{align}
Substitute \eqref{eq:big:flow} with \eqref{eq:big:flow1} to obtain the
Multi-Item Multi-Resource Capacitated Lot Sizing Problem with Setup
Times (\bit).

Obviously both can be applied simultaneously to obtain the Multi-Item Capacitated
Lot Sizing Problem with Setup Times (\bitres).

\section{Construction Heuristics}
Two basic approaches to construct solutions for the small- and
big-bucket models.

For simplicity we assume that the problem to solve is the \blsp. The
heuristics are adaptable to the \slsp{}

Both heuristics solve a series of sub-problems of the original
formulation. The sub-problems are also \NP-hard but due to their size they
are expected to be manageable with the use of general MIP heuristics
available in CPLEX, e.g., diving and Relaxation Induced Neighborhood
Search (RINS).

\subsection{Per Product Structure Level}
Given the product structure of the demanded items partition the
structures into sets given by their level in a given product
structure. In this case there are two approaches to divide the items:
(i) an item that occur on different levels in different production
structures can either be put in several sets according to the levels
or (ii) the item is put in the set given by the lowest level of the structures.

Next the a \bit{} for each set is solved starting from the top
down. The solution of on iteration is fixed before proceeding to the
next iteration. Due to initial stock settings a feasible solution can
always be obtained.

The algorithm is sketched as follows:
\begin{enumerate}
\item Divide items into sets each corresponding to a level in the
  product structures.
\item For each set of items sorted with top level items first
\item Solve a \bit{} for a given set of items.
\item Fix the solution and goto step 2.
\end{enumerate}

An obvious drawback of such a heuristic is the potential of producing
a lot of initial stock for the lower level items of the product
structure since these items are placed last. Hence, the production
plan for first time periods are not expected to be very good compared
to later time periods.

\subsection{Per Item Demand}
An alternative simple procedure is to plan the production of a product
structure one at a time. Two basic approaches are suggested: (i) plan the product structure for one
unit of end item demanded at a given time, or (ii) plan the product
structure for the entire demand of end item at a given time. If (i) is
used the end items should be planned with some kind of randomization
to diversify the production. If (ii) is used the end items can be
planned in decreasing order of time for the demand.

Planning a the product structure of an end item is done by solving a
\blsp. The problem is easier than the original problem since only the
demand of one time period is considered. However, if the product
structure is very deep, we suggest to split product structure into
smaller subtrees and solve them top down.

The algorithm is sketched as follows:
\begin{enumerate}
\item Produce a list of end items for planning
\item For each end item in list
\item Solve a \blsp{} for the demand of the end item
\item Fix the solution and goto step 2.
\end{enumerate}

The drawback with this heuristic is the possibility of high initial
stock for the end items being planned late in the procedure. This is,
due to capacity restrictions and consumed capacity by the production
of the first planned product structures.

\section{Uncertainty}
In the classical stochastic approach the problem is decomposed using
Benders Decomposition and the numerous scenarios (samples) are solved
as Benders subproblems. Examples of uncertain events are:
\begin{itemize}
\item Demand
\item Capacity
\item Lead-time
\end{itemize}
In principle, any constant in the above models can be considered under uncertainty.

\bibliographystyle{plainnat}
\bibliography{references}

\end{document}

%%% Local Variables:
%%% mode: latex
%%% TeX-master: t
%%% End:
