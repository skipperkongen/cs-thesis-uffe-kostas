\subsection{Litterature on Robust Solutions}

Martin: Vi har en plan for hvordan vi vil bruge litteraturen :)

We have found that finding robust solutions (in the sense defined
above) has been the subject of several and diverse published
articles. Some articles deal specifically with mixed integer problems,
while others look at more general problems. Below is a short review of
the articles, the introduced methodologies and the problems they deal
with.

\emph{Genaro J. Gutierrez} and \emph{Panagiotis Kouvelis} in their
article \emph{A robustness approach to international sourcing} discuss
the problem of finding robust solutions to the \emph{international
sourcing problem} \cite{kouvelis}. In this problem a firm needs to
select suppliers in various countries to support the demands of the
firm's international factory network. Robustness becomes an issue
because solutions to this problem depend on the realization of
uncertain parameters like the future exchange rate and macroeconomic
parameters. They present an algorithm for the problem which will find
the $N$ best robust solutions.

\emph{Lin}, \emph{Janak}, and \emph{Floudas} in their article \emph{A
new robust optimization approach for scheduling under uncertainty:
I. Bounded uncertainty} discuss the problem of finding robust
solutions to mixed integer problems in the domain of \emph{production
scheduling} in which objective function coefficients, left hand sides
and right hand sides are assumed to have a bounded uncertainty. By
means of auxiliary variables and constraints, they formulate a
deterministic robust counterpart problem to the stochastic problem. A
solution to the deterministic counterpart problem is also a robust
solution to the original stochastic problem \cite{lin} (NOTE: There
are some details that we need to understand better). We discuss a
similar topic in section \ref{sec:robust_subspace} on \emph{robust
subspaces} of stochasic mixed integer problem.

In their article \emph{Genetic Algorithms with a robust solution
searching scheme}, authors \emph{Tsutsui} and \emph{Ghosh} present a
genetic algorithm for finding solutions to problems that are sensitive
to small pertubations of the parameter values, by avoiding ``brittle''
solutions, i.e. solutions at sharp peaks \cite{tsutsui}. They point at
previous work of Sebald and Fogel \cite{sebald}, where fault tolerant
neural networks for pattern classification were designed by
explicitily causing random faults in the networks during training
using evolutionary programming. Tsutsui and Ghosh implement a similar
technique in their GA by adding noise to the phenotypic parameter
values of individuals before their fitness level is evaluated.


\subsection{Ben-tal 1}

% brugt 
A linear program (LP) formulation of a problem with data $(A,b)$ is a problem on the form

$min \lbrace{c^T x \vert  Ax\geq b\rbrace, x \in \mathbb{R}^n$.

% brugt
In robust optimization, convex optimization problems are studied for
which the data $(A,b)$ is not specified exactly and is only known to
belong to a given uncertainty set $\mathcal{U}$, yet for a solution
$x$ the constraints must hold for all possible values of the data from
$\mathcal{U}$, i.e. $Ax\geqb \forall(A,b) \in \mathcal{U}$. The
ensuing optimization problem is called robust optimization \cite{bental1}.

% brug i solving problems under uncertainty
An approach often used when solving an uncertain LP is to replace it
by an explicitly stated convex robust counterpart program (RC). If
certain assumptions can be made about the uncertainty set $U$, such an
RC is computationally tractable \cite{bental1}.

The problem $min \lbrace c^T x \vert Ax \geq b, \forall(A,b) \in
\mathcal{U} \lbrace}$ is such a robust counterpart (RC) of the uncertain LP
problem, and the vector $x^*$ solving equation REF is called a robust
solution of the uncertain problem \footnote{In section
\ref{sec:non_convex} we show that the set $\lbrace x \vert Ax\geqb
\forall(A,b) \in \mathcal{U} \rbrace$ is however not guaranteed to be convex.}. In our thesis 
we shall use a notion of robustness which is more akin to chance
constrained programming (see section \ref{sec:definition of
robustness}).

% brug i solving problems under uncertainty
Other approaches to robust optimization include \emph{stochastic
programming} (SP) \cite{steinwallace} (see section \ref{sec:stochastic_programming}),
\emph{scenario optimization} \cite{rockafellar} (see sections \ref{sec:scenario} and \ref{scenario:optimization}).

\subsection{Uncertain integer linear programs}

Many NP-hard problems can be formulated as integer linear programs
(IP), which are special cases of linear programs in which all decision
variables are required to be integer. A deterministic IP has the form
$min \lbrace{c^T x \vert Ax\geq b\rbrace$, where $x \in \mathbb{Z}^n$,
$A \in \mathcal{R}^{n,m}$, and $b \in mathbb{R}^n$.

An uncertain LP is similar to an uncertain IP with the only difference
that all decision variables are required to be integer, with data $(A,b)
\in \mathcal{U}$.

\subsection{Scenarios}

Given an instance of an uncertain IP $I$ with data $(A,b \in \mathcal{U})$, a
\emph{scenario} $S$ is a deterministic IP that has been derived from $I$ by replacing the 
data $(A,b) \in \mathcal{U}$ with the data $(A_S \in
\mathcal{R}^{n,m}, b_S \in \mathcal{R}^n)$. $(A_S, b_S)$ has been obtained by
sampling the uncertainty set $\mathcal{U}$. A scenario is thus a
specific outcome of the stochastic variables underlying $\mathcal{U}$.

\subsection{Definition of robustness}
\label{sec:definition of robustness}
In our thesis we'll consider uncertain integer linear programs
(uncertain IP) on the form $min \lbrace c^t x \vert P(Ax\geqb
\forall(A,b) \in
\mathcal{U}) \geq 95\% \rbrace$. 

We have thus relaxed the constraint that a solution $x$ must be
feasible for all realizations of $(A,b)$, to the constraint that a
solution $x$ must have probability of $95\%$ of being feasible given a
realization of $(A,b)$. This approach is similar to chance constrained
programming \cite{steinwallace}.

A key point in this thesis is that we
shall assume the underlying stochastic model of the data $(A,b) \in
\mathcal{U}$ to be known, namely that data is
\emph{normally distributed}.

\section{Til introduction}

The data $(A,b)$ associated with a linear program $min \lbrace c^T x
\vert Ax \geq b$ are ``uncertain" to some degree in most real world
problems \cite{bental}. In many models the uncertainty is ignored
altogether, and a representative nominal value of the data is used
(e.g. expected values).

A classical approach to deal with uncertainty is stochastic
programming (SP) (see e.g., [6,13] and references therein). But even
in this approach constraints may be violated, with certain penalty
(this is the case for SP with recourse [6,9], scenario optimization
\cite{rockafellar}, entropic penalty methods [1]) or with certain probability
(chance constraints). In the dominating penalty approach, even when
the random variables are degenerate (deterministic), the corresponding
SP model does not recover necessarily the original LP constraints, but
only a relaxation of these constraints. Thus, although this was not
stated explicitly in the past, SP treats in fact mainly soft
constraints. These remarks apply also to the recent scenario-based
penalty approach of Mulvey, Vanderbei and Zenios [11].



In this paper we study uncertainty associated with hard constraints,
i.e., those which must be satisfied whatever is the realization of the
data (A; b) within, of course, a reasonable prescribed ``uncertainty
set" U.  Feasibility of a vector x is thus interpreted as Ax>b forall(A; b)
in U: Consequently, we call the problem min{c^T x | Ax>b forall(A; b) in U}
the robust counterpart of the uncertain LP problem (1), and we
call a vector x solving Eq. (3) a robust solution of the uncertain
problem.


