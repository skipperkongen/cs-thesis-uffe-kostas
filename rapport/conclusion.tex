\chapter{Conclusion}
\label{sec:conclusion}
In this thesis we have presented the concept of partial robustness to
optimization problems. Through our discussions of problems under
uncertainty we have clarified which methods are used to solve problems
with robustness issues. We have also explained why the usual methods
are not applicable to finding solutions with partial robustness. In
the case of stochastic programming we have introduced it and discarded
it since it is a very analytical approach opposed to the very
experimental approach that we have taken.

We have described the two problems that we have been working with, the
knapsack problem and the lot-sizing problem, and how uncertainty can
be introduced to these problems in a reasonable manner.

Using simulation as a key to determining robustness we have presented
a method to finding robust solutions. This approach involves search
heuristics that use the robustness as an evaluation parameter. This is
why it has been important for us to use statistics to decrease the
number of simulations necessary to evaluate solutions. The use of
statistics and simulation has been a good combination, yielding fast
results and being easy to set up.

Through experimentation we have attempted to improve on the fat
solutions to our problems. Since the fat solution can be seen as an
adaptation of the methods used for robustness this has served as a
good comparison solution. For the knapsack problem we have obtained
good results using our approach, yielding an increase in profit
between 2 and 31$\%$, sometimes improving solutions that were already
very good to begin with. With these results we conclude that our
approach is applicable to the knapsack problem.

The lot-sizing problem proved to be a more challenging problem, as we
had also expected. Our approach with a genetic algorithm on the
knapsack problem proved to be worthless when dealing with the
lot-sizing problem. Just finding robust solutions was a challenge, but
using our concepts of simulations and scenario optimization we were
able to find solutions that were robust. In most cases the cost
increased only slightly indicating a good solution while in other
cases the cost increased more dramatically.

Summing up we believe our approach to be a good one when dealing with
simpler problems, since we are able to produce good results with a
reasonable amount of effort. For more complex problems the results are
more scattered and our approach has less chance of producing a useable
result.