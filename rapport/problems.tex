\chapter{Problems}
\label{sec:problem}

In order to create our framework for finding robust solutions we need
to work on some problems. We have chosen two problems to work on, the
knapsack problem and the capacitated lot-sizing problem. These two problems are
both optimization problems that can be modeled using integer
programming and thus fit perfectly into what we have discussed
earlier. The reason for choosing two problems is that the knapsack
problem has some very well established solution methods and problem
instances that we can use to show that our approach actually
works. The only problem with the knapsack problem is that it is
perhaps \emph{too} well described in the litterature and thus not much
interesting can be gained from our treatment of the subject.

This is why we have then applied our developed framework to the
lot-sizing problem, in order to find a field where our framework might
have some value in real life. 

%Production planning, of which the
%lot-sizing problem is a part, is used in many real-life instances
%where many stochastic factors are at work. It is thus easier for us to
%create realistic and understandable examples than what we feel capable
%of with the knapsack problem.

We will now present the two problems as well as the stochastic factors
that can be introduced. We will also specify which stochastic factors
we will be working with and which we choose to omit for simplicity
reasons.

\section{Knapsack problem}
\label{sec:knapsack_problem}

The easiest way to describe the knapsack problem is by example
\begin{example}
Imagine a robber who is robbing an art exhibit. The robber has brought
only one bag with him (the knapsack) and with him he has a list of how
much each item in the art exhibit is worth and how much space it will
take up in the knapsack. The object of the robber is to fill the
knapsack with the objects that will give him the largest amount of
profit.
\end{example}
Although optimizing criminal activity may not be a good career move
for us, this problem is nevertheless a very well studied problem and
actually has applications outside the world of crime (whew).

The problem can be formulated as an integer programming model:
{\setlength\arraycolsep{2pt}
\begin{eqnarray}\label{knapsack_definition}
\max & & \sum_{j=1}^n p_j x_j\\
\textrm{s.t. }& & \sum_{j=1}^n w_j x_j \leq c \\
& & x_j \in \{0,1\}, j=1,\dots,n
\end{eqnarray}}

In this model the parameters are follows:
\begin{align}
\begin{tabular}{cl}\label{knapsack_variables}
$p_j$ & The profit associated with item $j$.\\
$w_j$ & The weight associated with item $j$.\\
$c$   & The total capacity of the knapsack.
\end{tabular}
\end{align}
And the variables are:
\begin{align}
\begin{tabular}{cl}\label{knapsack_variables}
$x_j$ & 1 if item $j$ is added to the knapsack and 0 otherwise.\\
\end{tabular}
\end{align}
When we consider the problem as a stochastic one, then the constants
$p_j, w_j$ and $c$ are the stochastic elements. The validity of this
can be explained in real-life terms as follows. The profit of an item
may be decided by a free market in which the price fluctuates, thus a
stochastic $p_j$. If the weight of an item is based on some estimate,
then the actual weight of the item when it is to be packed may differ
from the one estimated, and so it makes sense to use a stochastic
$w_j$. The exact size of the knapsack may not be something we
control, and so it too may have a different size than what was
planned, making a stochastic $c$ plausible.

In our work with robustness we have chosen to leave the profit, $p_j$
as a deterministic constant, since the stochasticity of $p_j$ doesn't
affect the robustness of a solution but only the objective value of
the solution.


\section{Production Planning}
\label{sec:production_planning}
The second problem that we are trying to find robust solutions for is
the capacitated lot-sizing problem \cite{wolsey}. This is a problem in the field
of production planning, and after we have briefly
introduced production planning, we will give a thorough introduction to
lot-sizing problems and the many forms they can take in section \ref{sec:lot_sizing_problem}.

Production planning problems are a branch of optimization problems
specifically targeted at production planning. Thus in production
planning problems we try to optimize production outputs, production
cost, facility utilization, employee utilization, warehouse usage or
other factors involved in production. Each factor based on the knowledge we
have before production starts.

Depending on the problem being optimized, the time horizon we are
planning for will vary. Common for most problems is that the time
horizon will be divided into a number of smaller time intervals known
as buckets. A solution to the planning problem will yield a production
scheme for each bucket.

There are many different characteristics that define a specific
production planning problem. Here we will describe the four most
important characeristics with relation to the problem we will be working
on.

\subsection{Resources}
\label{sec:resources}

For production to occur something must exist which can transform raw
materials (or unfinished end-products) into finished
end-products. This can be one of many things e.g. people, machinery,
tools or budget if we view the problem more abstractly. No matter what
we are using to produce our end-products we shall refer to the
producer as a resource.

Problems are typically divided into single- or multi-resource
problems, which allows for production of one or more types of
end-products in each bucket.

\subsection{Capacity}
\label{sec:capacity}
When considering \emph{what} to produce in each bucket, that is what
the resources are set to produce, we also consider \emph{how much} is to be
produced in each bucket. In some models this amount can be set
to anything we desire (uncapacitated) or there may be a limit to the
amount each resource can produce (capacitated).

\begin{example} 
As an example of uncapacitated production consider an assembly of
tables that is done by untrained labourers, hired on a day-to-day
basis. We can (almost) always hire more labourers if we need to
assemble more tables, or hire fewer labourers if we need to assemble
fewer tables. In practice the amount of production in this example
will be uncapacitated.
\end{example}
Examples of capacitated problems are plentiful, just imagine a factory
with a set amount of machines. Getting a new machine takes a long time
and so the factory will be restricted (capacitated) in production by
the number of machines that they own.

\subsection{Items}
\label{sec:items}
Production planning problems can also deal with either one or more
end-products. In many models each of the resources can then produce
several or all of the end-products, but naturally with different cost
for each item. Whether a resource can produce just one or several
types of end-products in a bucket is dependant on the model and,
eventually, how the real world works of course.

\subsection{Levels}
\label{sec:levels}
The final characteristic that is important to our specific production
planning problem is whether the problem is single- or multi-level. 
In a single-level problem products are produced (using resources) from
raw materials that are available from external suppliers. 
\begin{example}
This could be the case if we manufactured flour made from wheat. The
wheat is available from farmers or wholesalers and we do not produce
anything from the wheat but merely sell it to stores or wholesalers.
\end{example}
In a multi-level problem each end-product has a bill of materials that
states which items are required to build which items. Thus each
end-product can possibly be required in another end-product (no cyclic
dependencies of course).
\begin{example}
As an example consider a factory manufacturing cars. To produce a car
several items must be available (chassis, engine, wheels etc.).  Each
of these items must also be either produced at the factory or bought
from external suppliers. If these items are produced at the factory
they will most likely also have subcomponents that need to be either
produced or purchased. Figure \ref{fig:car_factory} provides a small
illustration of the example.  \figur{Illustration of production flow
in a car factory}{fig:car_factory}{figurer/car_factory.eps}
\end{example}

There exist many other factors that depend on what production planning
problem is being solved but the four we have just presented are the
ones that are most pertinent to the problem we will be dealing with,
which will be described in the following chapter.

\section{Lot-sizing}
\label{sec:lot_sizing_problem}
One production planning problem is the capacitated lot-sizing problem
which models a manufacturer trying to minimize the cost of storage and
production of items, while meeting demands for those items from
costumers at the time of the demand. Other production planning
problems might focus on maximizing output or utilizing all resources.

One assumption in a lot-sizing problem is that there is some penalty to be paid if
a demand is not met, this could model the real-life situation of
having a contract with a customer for which there is a penalty if a
deadline is missed. It could also model the fact that customers are
lost if delivery is not reliable. Sometimes this assumption is
circumvented in the model by disallowing the situation of a demand not
being met.

Another assumption is that keeping products in storage incurs a
cost. This can be viewed as an actual storing cost (cost of
warehouses) but more realistically it should be viewed as the lost
potential income from having capital tied up in the stocked items.

Optimally we will have an inventory of zero and only produce
end-products at the time they are demanded by the customer, but this
is not always a feasible strategy. Since the problem is capacitated
(see section \ref{sec:capacity}) it may be necessary to produce items
in advance since a full order may not be producable within the same
timeslot that it is demanded.

Within lot-sizing problems there are many different variations, primarily from
choosing different settings of the four factors just described in
section \ref{sec:production_planning}

We will now sketch some of the problems here:

\begin{description}

\item[Capacitated lot-sizing problem] In the capacitated lot-sizing
problem the production of items consumes capacity on resources,
with an upper limit existing on the capacity of each resource.

Each item produced consumes a share of the available capacity on a
resource. This creates a limit on the number of items that can
be scheduled for production on a single resource during a time period.

One can also model setup times for the resources as a capacity drain,
such that the production of an item consumes an additional fixed
amount of capacity on the resource, normally only during the first
time period of production.

\item[Multi-resource lot-sizing problem] In lot-sizing the production of
items may be assigned to a single resource, or to several
resources. In the latter case we speak of the multi-resource
lot-sizing problem. In a real-life instance
where there are several resources, we may still choose to model the
situation as a single resource problem if all the resources are
capable of the same in matters of production speed and products to be
processed. 
\begin{example}
In a shoe factory with 500 workers, the workers may be viewed as one
resource with 500 times the capacity of a single worker. 
\end{example}
This can be done if the production of items is fast and the buckets we
are planning for are sufficiently large.

\item[Multi-level/Multi-item/Single-item] Multi-level lot-sizing problem,
 is the case where the production of an item consumes other
items as part of the production process. Such dependencies among items
form a tree refered to as the bill of materials as described in
section \ref{sec:items}.

If several items are produced, but no items are required for the
production of other items (i.e. the bill of materials consists of only
leaves), the problem is refered to as multi-item lot-sizing problem.

If only a single item is produced the problem is refered to simply as
lot-sizing.
\end{description}

The mathematical formulation of a lot-sizing problem depends on the
specific problem that is being treated. In our thesis we will be
dealing with the multi-item capacitated lot-sizing problem since it is
sufficiently complex in respect to our goal of finding robust
solutions to hard problems. There are plenty of stochastic parameters to tweak in this
problem as we will describe in the next section.


\subsection{Modeling the capacitated lot-sizing problem}
\label{sec:modelling_clsp}

In this chapter we will present the mathematical formulation of the multi-item
capacitated lot-sizing problem and explain the meaning of the individual variables and constants.

The model looks like this
{\setlength\arraycolsep{2pt}
\begin{eqnarray}\label{clsp_definition}
\min  \sum_{i\in items}^{}&& \left( h_0^i s_0^i + \sum_{t\in T} (
h_t^i s_t^i + p_t^i x_t^i) \right) \\
\textrm{s.t. } s_{t-1}^i + x_{t}^i &=& d_t^i + s_{t}^i, \textrm{ }t
\in T,\ i \in items \\
\alpha_t^i x_t^i + \beta_t^i z_t^i &\leq& c_t y_t^i, \textrm{ }t \in
T, \ i \in items\\
\sum_{i \in items} y_t^i &\leq& 1,\textrm{ }t \in T, \ i \in items\\
x_t^i \geq 0,\ s_t^i \geq 0&,&\ y_t^i, z_t^i \in \mathbb{B}
\end{eqnarray}}

Here the variables are
\begin{itemize}
\item $s_t^i$ is the number of units of item $i$ on stock at time $t$
\item $x_t^i$ is the number of units of item $i$ being produced at time $t$
\item $z_t^i$ indicates if production is being set up at time $t$
\item $y_t^i$ indicates if production of item $i$ can occur at time $t$
\end{itemize}

The constants are
\begin{itemize}
\item $h_t^i$ is the inventory cost at time $t$ for one unit of item $i$
\item $p_t^i$ is the production cost at time $t$ for one unit of item $i$
\item $d_t^i$ is the demand at time $t$ for item $i$
\item $c_t$ is the capacity of production at time $t$
\item $\alpha_t^i$ is the capacity consumed by producing one unit of
  item $i$ at time $t$
\item $\beta_t^i$ is the capacity consumed by setting up for
  production of item $i$ at time $t$.
\end{itemize}

The set $T$ indicates all the time intervals (buckets) that we are
planning for. The set $items$ contains all the items that can be produced.

\subsection{Uncertainty in multi-item capacitated lot-sizing}
As with the knapsack problem any constant in our model of capacitated
lot-sizing problem can be stochastic. We
will now describe briefly how this could be a reflection of a
real-world situation.

\begin{itemize}
\item Demand, $d_t$, being stochastic is easy to imagine if the demand
  is from a free market, and not some long-term contract. Many
  industries are subject to this kind of demand and often experience
  fluctuations in demand.
\item Capacity, $c_t$, can also vary stochastically. If the resource
  being used are people, then the fluctuations could represent how
  many people are sick, or that the entire group of people may be less
  productive at one time than at another. Other factors that could
  make capacity vary are products that must be discarded. They have
  used up capacity but do not ontribute to satisfying demand.
\item Setup times, $\beta^i _t$, may have stochastic duration. e.g. as
a consequence of differences in skill level of the worker setting up a
resource for production in a factory. This would lead to a smaller or
bigger waste of capacity depending on the skill of the worker.
\item Inventory cost, $h_t$, can also very easily be stochastic since
  it is the ``cost'' incurred by having money tied up in inventory
  this value will rise and fall with the current interest rates in
  real-life.
\item Production cost, $p_t$, could be stochastic if for example a
  varying amount of end-product is produced from the same amount of
  raw material. Here the discarded products will also influence the
  constant, since a discarded end-product will still have consumed some
  amount of raw material.
\end{itemize}

