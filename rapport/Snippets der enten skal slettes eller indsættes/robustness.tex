\section{Robust solutions}
\label{sec:robust_solutions}

Our objective is to find a solution $z$ to an instance $I$ of the
version of MLMR-CLSP described in section
\ref{sec:sec:extended_problem_formulation} that minimizes cost and is
95\% certain to succeed (be feasible) given the assumption that a
selected subset $Q$ of the constants involved in constraints (BLA and
BLA) are stochastic.

When refering to MLMR-CLSP, we shall mean the version described in
section \ref{sec:sec:extended_problem_formulation} unless otherwise
stated.

\subsection{Modelling a Stochastic System}
\label{sec:distributions}
We'll associate either a \emph{normal}, \emph{weibull} or \emph{gamma}
distributions with each constant in $Q$. Furthermore we'll associate \emph{binomial} distributions 
with a subset of the constants of $Q$, e.g. with the time-dependent capacity $C^k_t$ of resource $k$.


\subsection{Methods for finding robust solutions}
\label{sec:alternative_methods}

The following is an overview of other methods that have been used to
achieve robustness in production planning. In section
\ref{sec:our_approach} we'll describe the method we use in this
thesis.

\begin{itemize}

\item \emph{Stochastic programming} and more precisely the variation
called \emph{chance constrained-programming}. These techniques are
presented in section \ref{sec:stochastic_programming}

\item Employing \emph{buffers} on constraints. If a constraint exists
for variable $x \geq b$ for $x, b \in \mathbb{R}$ in the original
formulation, employing a buffer means $x \geq b + \epsilon_1$. This
ensures that $x$ does not assume a value on the boundary of the
solution space, but it is reasonable to assume that such an approach
if used without caution would results in socalled ``fat'' solutions
\cite{kallwallace} \cite{nogetsimonsagde}.
\end{itemize}

\subsection{Our approach}
\label{sec:our_approach}
In our approach we are provided with a black-box deterministic solver
to MLMR-CLSP. This solver is known to use a heuristic
large-neighbourhood search and furthermore runs on the assumption that
all constants can be known a priori.

\subsection{Finding robustness solutions}
\label{sec:estimating_robustness}
How do you generate production schedules with a expected high
real-life robustness using a deterministic solver which uses a
completely static model of the real world? 

Such a deterministic solver can not take into account many different
scenarios. The virtue of such a solver is that it is able to quickly
find good solutions to the problem if constants are assumed to be
known a priori, which is abolutely crucial because MLMR-CLSP is a very
hard problem!

An algorithm which can assume that all constants are completely static
is likely to be far more efficient than an algorithm which has to
account for even a small degree of stochastic freedom in each
constant. By that argument we rule out the option for finding an
algorithm of the latter type for this thesis, while reminding the
reader about the alternative methods mentioned in section
\ref{sec:alternative_methods}.

It should be noted that it is not a viable approach to simply increase
the level of detail of the model. A modeller might try using
fine-grained time-dependent constants to model variation in production
costs over time, or the capacities of resources over time, to account
for phenomena such as \emph{learning-tiredness} \cite{digiesi}. The
problem remains however that even the most detailed static model
assumes that all constants are known \emph{a priori} and can
consequently not account for the \emph{unexpected}. No matter how
fine-grained a static model is, it models a problem for exactly one
scenario.


\subsubsection{Estimating robustness of solutions}
\label{sec:estimating_robustness}

We want to estimate the robustness of a solution $z$ to an instance
$I$ of MLMR-CLSP under the assumption that a selected set of constants
$Q$ are stochastic. The set $D$ is a set of distributions that
describes these stochastic variables.

For this thesis we will estimate the robustness of a solution by performing a large number
of experiments, where each experiment consists of generating a
\emph{scenario} and testing the feasibility of the solution $z$ in that scenario.

Each experiment is conducted in the follow straight-forward manner:

\begin{itemize}

\item For each constant in $q \in Q$ we generate a sample $\xi$ by
sampling the distribution $d \in D$ associated with $q$.

\item For each generated sample $\xi$, we substitute the value $\xi$
for the constant $q$ in $I$ yielding a new instance $I'$ called a
\emph{scenario}.

\item The feasibility of $z$ is tested with regard to the scenario
$I'$, recalling that modifying the value of constants in $I$ will
have yielded constraints in $I'$ that are different from the
constraints of the original instance $I$, and will possibly (but not necessarily) have
affected the solution space. Feasibility is tested simply by testing
the solution $z$ against all affected constraints in $I'$.

\end{itemize}

We require a solution to be feasible in at least $95\%$ of the generated scenarios in order to be labelled robust.

UFFE INDS�T NOGET OM CONFIDENS INTERVALLER HER.

\subsection{What shall we do with the drunken sailor?}
\label{sec:search_for_robust_solutions}
In section \ref{sec:estimating_robustness} we described how we intend
to estimate the robustness of a solution $z$. Assuming that a
particular solution $z$ was found to be non-robust and assuming that
$z$ was generated by a deterministic solver, it does not make sense to
ask the solver for another solution, as it will provide the exact same
solution given the same parameters.

Two options seem to present themselves at this point. 

\begin{enumerate}

\item Alter the parameters and the problem instance provided to the
deterministic solver so that it finds a solution which is more
robust. This is discussed in section \ref{sec:altering_the_instance}.

\item Pass the solution to another algorithm specialized in searching
for robust solutions when provided with optimal but non-robust
ones. This is discussed in section
\ref{sec:genetic_algorithm_on_solutions}

\end{enumerate}

We will discuss both these xoptions in this thesis.

A third option that seems obvious is to construct an algorithm that is
efficient at finding solutions that are optimal while subject to
requirements regarding robustness. MLMR-CLSP however, is a very hard
problem, which requires a very efficient algorithm to solve. 

An algorithm which can assume that all constants are completely static
is likely to be far more efficient than an algorithm which has to
account for even a small degree of stochastic freedom in each
constant. By that argument we rule out the option for finding such an
algorithm for this thesis, while reminding the reader about the
alternative methods mentioned in section
\ref{sec:alternative_methods}. REPEAT

\subsubsection{Pipe dream 1: Altering the instance}
\label{sec:altering_the_instance}

An approach which we will discuss is how to coax a deterministic
solver into providing robust solutions by modifying the instances
provided to the solver. In this approach the deterministic solver
computes a solution $z_j$ to an instance $I_j$ of MLMR-CLSP. If this
solution is non-robust, a new instance $I_{j+1}$ is generated from $I_{j}$ by changing selected constants of $I_j$. 

Let $\mathcal{L}(I)$ mean the solution space of instance $I$, then
$\mathcal{L}(I_j) \subseteq \mathcal{L}{I_0}$ must hold for all $j$. This is called the subspace property.

We'll iteratively generate new instances $\lbrace I_0, I_1, ... I_j$
until a solution $z_j$ is found to $I_j$ which is estimated to be
\emph{robust}. By ensuring the subspace property $z_j$ is guaranteed to be feasible with regard to the original
instance $I_0$. 

As mentioned the shifting of constants which transforms $I_j$ into
$I_{j+1}$ must result in an instance where $\mathcal{L}(I_j+1)
\subseteq \mathcal{L}(I)$, otherwise a solution to $I_{j+1}$ could not
be guaranteed to be feasible for instance $I$. Furthermore the
objective of the constraining the solutionspace is to force the
deterministic solver to find a more robust solution.



