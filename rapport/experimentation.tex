\chapter{Experimentation and Results}
\label{sec:experimentation_and_results}

In this chapter we will be describing the different experiments we
have conducted and the results that we have gathered. The primary
objective of this is of course to see if our approach with a genetic
algorithm can produce robust solutions with good objective values. 

We can easily find out if the solutions are robust using the methods
we described in chapter \ref{sec:testing_robustness}. It is a
different matter when we try to check if we obtain good objecive
values. The problem is that we do not have data to see how good robust
solutions have been found on the data that we are working with. This
is not as big a problem as it might sound. Since our main goal is to
see if our approach can provide some useful insights, we will simply
measure our solutions against those that we can easily obtain. That
is, we will be comparing our solutions with the fat solutions
described in chapter \ref{sec:fat_solutions} as well as the solutions
to the problems with no stochasticity.

We will treat the experimentation of the knapsack problem and the
lot-sizing problem seperately since the results vary greatly and the
approach is somewhat different although there are significant overlaps.

\section{Knapsack problem experimentation}
\label{sec:experiments_kp}
The primary object of our experimentation is to find out if we can
generate a better solution than the fat solution (see chapter
\ref{sec:fat_solutions}). We also wish to test if we obtain roughly the
same results when we run our algorithm several times on the same
problem as this has some impact on the usefulness of our approach.

Since the fat solution is our naive bid at a robust solution, an
improvement over this is imperative. We will also be comparing our
solutions to solving the deterministic knapsack problem where we
substitute the stochastic variables with their respective mean
values. This will generally lead to solutions that have a robustness
of $0.5$ and these solutions will only serve as a reference to how
much it costs to obtain the additional $0.45$ points of robustness. We
have no way of proving that the optimal solution to the deterministic
knapsack problem is an upper bound on the robust solution to the
stochastic knapsack problem but it is resonable to think that no
robust solution can be found that will improve on the deterministic
knapsack problem.

\subsection{Generated instances}
\label{sec:generated_instances_knapsack}
For the experiments we have generated a lot of different stochastic
instances. The different instances have been generated to help us
identify which parts of the stochastic knapsack problem has the
greatest influence on the behaviour of our algorithm.

The following list contains the parameters of the stochastic knapsack
problem that have been augmented from instance to instance and the
different values they assume.
\begin{itemize}
\item Number of items (50 or 200)
\item Stochastic capacity (yes or no)
\item Standard deviation of capacity (high or low)
\item Stochastic weight (yes or no)
\item Standard deviation of weight  (high or low)
\item Seeding the algorithm (yes or no)
\end{itemize}
Not all combinations of these values make sense, i.e. having high
standard deviation of the weights when they are not stochastic seems
silly. Naturally we have not experimented with any of the non-sensical
combinations but apart from those we have only deselected the
completely non-stochastic combinations. This still leaves us with 16
instances with different numerical values, we will be running our
algorithm on each of these instances twice, once using seeding and
once without. The instances are presented in table
\ref{tab:kp_instances}. The instances have been named A-P for easy
identification. In the remainder of the tables we will be sorting the
tables according to the information we are analyzing, thus the
individual instances will appear at different places in the tables, in
this respect the names should help in finding a particular instance.
This should not be necessary since the individual instances are rather
uninteresting and the trends that can be spotted are more interesting.

\begin{align}
\label{tab:kp_instances}
\begin{tabular}{llll}
 \hline 
Name&Items&Weight Deviation&Capacity Deviation\\ \hline 
A&50&None&Low\\ \hline 
B&50&None&High\\ \hline 
C&50&Low&None\\ \hline 
D&50&Low&Low\\ \hline 
E&50&Low&High\\ \hline 
F&50&High&None\\ \hline 
G&50&High&Low\\ \hline 
H&50&High&High\\ \hline 
I&200&None&Low\\ \hline 
J&200&None&High\\ \hline 
K&200&Low&None\\ \hline 
L&200&Low&Low\\ \hline 
M&200&Low&High\\ \hline 
N&200&High&None\\ \hline 
O&200&High&Low\\ \hline 
P&200&High&High\\ \hline 
\end{tabular}
\end{align}

\subsubsection{Construction of items}
\label{sec:construction_of_items}
In constructing the items (profits and weights) for the instances, we
have chosen to follow the same method as described in \cite{bertsimas}
where the items in the generated instance have a mean weight randomly
selected from the interval $[20,29]$ and a profit from the interval
$[16,77]$. These mean values are evenly distributed along their
respective intervals, meaning there are (on average) as many items
with weight 20 as there are items with weight 24.

The weights of each instance have been assigned a standard deviation
dependent on whether we wanted a high concentration of values aound
the mean or a more even spread. More specifically the deviation of the
weight of an item is set so 95$\%$ of the sampled weights will fall in
an interval that extends 10$\%$ to either side of the mean (low
deviation) or up to 50$\%$ to either side (high deviation). As an
example of this we have included figure \ref{fig:low_deviation} to
show the distribution of weights for an item with mean weight 20 and a
low deviation and on figure \ref{fig:high_deviation} we have the same
item with a high deviation.

\figur{Distribution of weight for an item with weight 20 and a low
  standard deviation.}{fig:low_deviation}{figurer/normal_20_1.eps}

\figur{Distribution of weight for an item with weight 20 and a high
standard deviation.}{fig:high_deviation}{figurer/normal_20_5.eps}


We have included the complete data for instance E in appendix
\ref{app:kp_scenario} as a reference to how the instances are
built. Since the instances take up a lot of pages when printed, we
have not included the remaining 15, if you are interested in seeing
them, please contact us by e-mail.

\subsection{Observations on the fat solutions}
In table \ref{tab:kp_fat_solutions} we show the
deterministic and fat solutions to the 16 instances, described in
section \ref{sec:generated_instances_knapsack}. In addition to the
values of the solutions we also show the ratio $\frac{\text{fat
  solution}}{\text{deterministic solution}}$. The table has been
sorted according to the value of this ratio (lowest to highest).

\begin{align}
\label{tab:kp_fat_solutions}
\begin{tabular}{lllllll}
\hline 
Name &Items&$\textbf{Weight}$   &$\textbf{Capacity}$ &Deterministic&Fat     &\multicolumn{1}{c}{$\textbf{Fat}$}\\
Name &     &$\textbf{Deviation}$&$\textbf{Deviation}$&Solution
&Solution&\multicolumn{1}{c}{$\overline{\textbf{Deterministic}}$}\\
\hline
H&50 &$\textbf{High}$&$\textbf{High}$&2240&1171&$\textbf{0.523}$\\ \hline 
P&200&$\textbf{High}$&$\textbf{High}$&8600&4777&$\textbf{0.555}$\\ \hline 
M&200&$\textbf{Low}$&$\textbf{High}$&8763&5832&$\textbf{0.666}$\\ \hline 
E&50 &$\textbf{Low}$&$\textbf{High}$&2289&1534&$\textbf{0.670}$\\ \hline 
B&50 &$\textbf{None}$&$\textbf{High}$&2126&1493&$\textbf{0.702}$\\ \hline 
J&200&$\textbf{None}$&$\textbf{High}$&8314&5841&$\textbf{0.703}$\\ \hline 
G&50 &$\textbf{High}$&$\textbf{Low}$&2224&1676&$\textbf{0.754}$\\ \hline 
O&200&$\textbf{High}$&$\textbf{Low}$&8312&6406&$\textbf{0.771}$\\ \hline 
F&50 &$\textbf{High}$&$\textbf{None}$&2156&1739&$\textbf{0.807}$\\ \hline 
N&200&$\textbf{High}$&$\textbf{None}$&8738&7150&$\textbf{0.818}$\\ \hline 
L&200&$\textbf{Low}$&$\textbf{Low}$&8444&7775&$\textbf{0.921}$\\ \hline 
D&50 &$\textbf{Low}$&$\textbf{Low}$&2233&2082&$\textbf{0.932}$\\ \hline 
I&200&$\textbf{None}$&$\textbf{Low}$&8379&7983&$\textbf{0.953}$\\ \hline 
A&50 &$\textbf{None}$&$\textbf{Low}$&2298&2197&$\textbf{0.956}$\\ \hline 
C&50 &$\textbf{Low}$&$\textbf{None}$&2305&2227&$\textbf{0.966}$\\ \hline 
K&200&$\textbf{Low}$&$\textbf{None}$&8277&8021&$\textbf{0.969}$\\ \hline 
\end{tabular}
\end{align}

As can be seen from the table the fat solutions vary from being only
half as good as the deterministic solutions to being close to the
deterministic solutions. The good results for the fat solutions are
obtained when only weights or capacity is stochastic and when
the stochastic elements have a low standard deviation. Conversely the
bad results come from high levels of stochasticity. Since the quality
of the fat solutions vary so much, we will include the measure
$\frac{\text{Fat}}{\text{Deterministic}}$ in most of the rest of the results
that compare our solutions to the fat solution.

\subsection{Effect of seeding}
\label{sec:effect_of_seeding}
The first ting we will analyze with regards to our algorithm is the
effect of seeding. In the following table we have listed the results
of running our algorithm on the 16 instances, with and without using
seeds. As can be seen from table \ref{tab:seeds} the algorithm always
performs better when using seeds. With this result in hand we will
omit the results of running without seeds from the remainder of the
experiments.

It is no surprise that running the algorithm with seeds generates
better results than without seeds. With seeds we start off with
solutions that are optimal, given a specific scenario. This optimal
solution may not be robust, but it gives us a strong starting point
from which to find solutions that are robust.


\begin{align}
\label{tab:seeds}
\begin{tabular}{lllllll}
 \hline
Name&$\textbf{Items}$&Weight   &Capacity &Sol. value & Sol. Value
&\multicolumn{1}{c}{$\textbf{No seeds}$}\\
&     &Deviation&Deviation&no seeds
 &seeds&\multicolumn{1}{c}{$\overline{\textbf{Seeds}}$}\\ \hline
J&$\textbf{200}$&None&High&5224&5819&$\textbf{0.898}$\\ \hline 
M&$\textbf{200}$&Low&High&5401&5934&$\textbf{0.910}$\\ \hline 
P&$\textbf{200}$&High&High&5351&5802&$\textbf{0.922}$\\ \hline 
L&$\textbf{200}$&Low&Low&7575&8035&$\textbf{0.943}$\\ \hline 
K&$\textbf{200}$&Low&None&7814&8242&$\textbf{0.948}$\\ \hline 
N&$\textbf{200}$&High&None&8127&8473&$\textbf{0.959}$\\ \hline 
B&$\textbf{50}$&None&High&1514&1574&$\textbf{0.962}$\\ \hline 
O&$\textbf{200}$&High&Low&7435&7647&$\textbf{0.972}$\\ \hline 
I&$\textbf{200}$&None&Low&7637&7802&$\textbf{0.979}$\\ \hline 
H&$\textbf{50}$&High&High&1515&1540&$\textbf{0.984}$\\ \hline 
E&$\textbf{50}$&Low&High&1578&1602&$\textbf{0.985}$\\ \hline 
G&$\textbf{50}$&High&Low&2056&2084&$\textbf{0.987}$\\ \hline 
A&$\textbf{50}$&None&Low&2179&2197&$\textbf{0.992}$\\ \hline 
C&$\textbf{50}$&Low&None&2278&2296&$\textbf{0.992}$\\ \hline 
F&$\textbf{50}$&High&None&2069&2081&$\textbf{0.994}$\\ \hline 
D&$\textbf{50}$&Low&Low&2135&2140&$\textbf{0.998}$\\ \hline 
\end{tabular}
\end{align}

\subsection{Comparing the algorithm to the fat solution}
In table \ref{tab:ga_vs_fat} we list the results of the genetic
algorithm versus the results of the fat
solution. As can be seen our algorithm doesn't achieve better results in all
cases. The three cases where our algorithm doesn't outperform the fat
solution all have one thing in common: only the capacity is
stochastic. This means that there is in fact only one stochastic
parameter in the problem, under those circumstances it isn't
surprising that the fat solution can't be outperformed, it is after
all an optimal solution to the knapsack problem with a reduced
capacity and the genetic algorithm hasn't much hope of obtaining a
better solution. 

\begin{align}
\label{tab:ga_vs_fat}
\begin{tabular}{llllllll}
 \hline 
Name&Items&$\textbf{Weight}$   &Capacity &\multicolumn{1}{c}{Fat}
& GA         &\multicolumn{1}{c}{GA}&\multicolumn{1}{c}{$\textbf{GA}$}\\
    &     &$\textbf{Deviation}$&Deviation&\multicolumn{1}{c}{$\overline{\text{Deterministic}}$}& Sol. Value &\multicolumn{1}{c}{$\overline{\text{Deterministic}}$}&\multicolumn{1}{c}{$\overline{\textbf{Fat}}$}\\ \hline
I&200&$\textbf{None}$&Low&0.953&7802&0.931&$\textbf{0.978}$\\ \hline 
J&200&$\textbf{None}$&High&0.703&5819&0.700&$\textbf{0.996}$\\ \hline 
A&50&$\textbf{None}$&Low&0.956&2197&0.956&$\textbf{1.000}$\\ \hline 
M&200&$\textbf{Low}$&High&0.667&5934&0.677&$\textbf{1.015}$\\ \hline 
K&200&$\textbf{Low}$&None&0.969&8242&0.996&$\textbf{1.028}$\\ \hline 
D&50&$\textbf{Low}$&Low&0.932&2140&0.958&$\textbf{1.028}$\\ \hline 
C&50&$\textbf{Low}$&None&0.966&2296&0.996&$\textbf{1.031}$\\ \hline 
L&200&$\textbf{Low}$&Low&0.921&8035&0.952&$\textbf{1.033}$\\ \hline 
E&50&$\textbf{Low}$&High&0.666&1602&0.700&$\textbf{1.050}$\\ \hline 
B&50&$\textbf{None}$&High&0.702&1574&0.740&$\textbf{1.054}$\\ \hline 
N&200&$\textbf{High}$&None&0.819&8473&0.970&$\textbf{1.185}$\\ \hline 
O&200&$\textbf{High}$&Low&0.771&7647&0.920&$\textbf{1.194}$\\ \hline 
F&50&$\textbf{High}$&None&0.807&2081&0.965&$\textbf{1.197}$\\ \hline 
P&200&$\textbf{High}$&High&0.555&5802&0.675&$\textbf{1.215}$\\ \hline 
G&50&$\textbf{High}$&Low&0.753&2084&0.937&$\textbf{1.245}$\\ \hline 
H&50&$\textbf{High}$&High&0.523&1540&0.688&$\textbf{1.315}$\\ \hline 
\end{tabular}
\end{align}

From looking at the data it seems that we can say something in general
about the impact of the three parameters, defining each instance on the
quality of the solution generated by the genetic algorithm compared to
the fat solution.
\subsubsection{Stochastic weight}
\label{sec:kp_exp_stochastic_weight}
As can be seen from table \ref{tab:ga_vs_fat} there is a very clear
relationship between the deviation of the the weight of the items and
the value of $\frac{\text{GA solution}}{\text{fat solution}}$. If we
disregard the results for 50 items with no weight deviation
(deterministic weight) and high capacity deviation, we can divide the
results into three groups based on the weight deviation.
\begin{itemize}
\item $\textbf{No weight deviation}$ as remarked earlier, results in
  poorer results or just the same as the fat solution. The conclusion
  is that using the algorithm un such problems are, at best, a waste
  of time.
\item $\textbf{Low weight deviation}$ has some improvements over the
  fat solutions. The improvements vary from 1 to 5$\%$, with an
  average of $3.1\%$, which may not
  seem like a lot, but some of the fat solutions are within 96$\%$ of
  the deterministic solution, which makes an improvement of even 1
  percentage point look a lot better.
\item $\textbf{High weight deviation}$ achieves the best results and
  generate improvements between 18 and 31$\%$, with an average of
  $22.5\%$. The fat solutions we are comparing to are naturally also
  quite a lot worse than the ones being compared using low weight
  deviation.
\end{itemize}
This might indicate that using our algorithm on problems with high
weight deviation is the approach that will generate the best results
overall. However if we evaluate the results with a different
objective, then the picture is less clear. The first comparison was
how much we improved on the fat solution, but what if we instead look
at how much we decrease the difference to the deterministic solution?
We have already stated that the deterministic solution is not an upper
bound but it might be useful to use it as such in order to cast some
more light on the matter. In mathematical terms this is ����������������
If we present the solutions sorted according to how much they decrease
the gap to the determinist solution, we get table \ref{tab:ga_vs_fat_alt}

\begin{align}
\label{tab:ga_vs_fat_alt}
\begin{tabular}{llllll}
 \hline 
Name&Items&$\textbf{Weight}$   &$\textbf{Capacity}$
&\multicolumn{1}{c}{GA}
&$\textbf{Decrease in}$\\
    &     &$\textbf{Deviation}$&$\textbf{Deviation}$
&\multicolumn{1}{c}{$\overline{\text{Fat}}$}
&$\textbf{gap to Det. Sol.}$\\ \hline
I&200&$\textbf{None}$&$\textbf{Low}$&0.978&$\textbf{-0.450}$\\ \hline 
J&200&$\textbf{None}$&$\textbf{High}$&0.996&$\textbf{-0.010}$\\ \hline 
A&50&$\textbf{None}$&$\textbf{Low}$&1.000&$\textbf{0.000}$\\ \hline 
M&200&$\textbf{Low}$&$\textbf{High}$&1.015&$\textbf{0.031}$\\ \hline 
E&50&$\textbf{Low}$&$\textbf{High}$&1.050&$\textbf{0.101}$\\ \hline 
B&50&$\textbf{None}$&$\textbf{High}$&1.054&$\textbf{0.128}$\\ \hline 
P&200&$\textbf{High}$&$\textbf{High}$&1.215&$\textbf{0.268}$\\ \hline 
H&50&$\textbf{High}$&$\textbf{High}$&1.315&$\textbf{0.345}$\\ \hline 
D&50&$\textbf{Low}$&$\textbf{Low}$&1.028&$\textbf{0.384}$\\ \hline 
L&200&$\textbf{Low}$&$\textbf{Low}$&1.033&$\textbf{0.387}$\\ \hline 
O&200&$\textbf{High}$&$\textbf{Low}$&1.194&$\textbf{0.651}$\\ \hline 
G&50&$\textbf{High}$&$\textbf{Low}$&1.245&$\textbf{0.745}$\\ \hline 
F&50&$\textbf{High}$&$\textbf{None}$&1.197&$\textbf{0.820}$\\ \hline 
N&200&$\textbf{High}$&$\textbf{None}$&1.185&$\textbf{0.833}$\\ \hline 
K&200&$\textbf{Low}$&$\textbf{None}$&1.028&$\textbf{0.863}$\\ \hline 
C&50&$\textbf{Low}$&$\textbf{None}$&1.031&$\textbf{0.885}$\\ \hline 
\end{tabular}
\end{align}

According to table \ref{tab:ga_vs_fat_alt} the relationship between
improvements on the fat solution and the deviation of the weight is
not as clear as we concluded from the numbers in table
\ref{tab:ga_vs_fat}.  Although we still see that a deterministic
weight leads to the algorithm performing worse than the fat solution,
we now have a more mixed picture of how low or high deviation affects
the improvement since the highest improvements are now obtained on
instances with low weight deviation. Generally we will still say that
the high deviation allows the algorithm to deliver the best results as
the improvements on these instances are $61\%$ on average whereas
they are only $44.2\%$ for the instances with low weight deviation.

\subsubsection{Number of items}
\label{sec:kp_exp_number_of_items}
Looking at table \ref{tab:ga_vs_fat} and \ref{tab:ga_vs_fat_alt} there
is no clear pattern regarding the number of items in the
instances. When calculating the averages of the performance of the
genetic algorithm versus the fat solution we arrive at $11.5\%$ for
the instances with 50 items and $8\%$ for the instances with 200
items, indicating that the algorithm performs better on smaller
instances. Calculating the averages of improvement in distance to the
deterministic solution gives us $42.6\%$ improvement for instances
with 50 items and $32.1\%$ improvement for the instances with 200
items, again indicative of the algorithm performing better on smaller
instances.

That the algorithm performs better on smaller instances may be
connected to the fact that we are dealing with a smaller search
space. With only 50 items there are no more than $2^{50} \approx
1000^{5}$ possible combinations of which items to bring. Another
explanation may be that more items, and larger capacity may also
decrease the impact of the discrete nature of the problem, since the
inability to bring a single item is reduced in impact by a factor of 4.

\subsubsection{Stochastic capacity}
\label{sec:kp_exp_stochastic_capacity}
Looking at how our algorithm performs with respect to the deviation of
the capacity of the knapsack we find the opposite relationship of the
weights, namely that high deviation leads to less improvement by the
algorithm. The average improvements for the three types of deviation
are:
\begin{description}
\item[High deviation:] $14.4\%$
\item[Low deviation:]  $28.6\%$
\item[Deterministic capacity:]  $85.0\%$
\end{description}

\subsection{Repeated runs and number of generations}
We have run the algorithm 10 times using seeding on instance E, as
with the experiments shown so far we have used 100 generations. The
purpose was to see if we could benefit from running the algorithm
several times or perhaps in parallel. The result is shown in table
\ref{tab:ga_repeat}

\begin{align}
\label{tab:ga_repeat}
\begin{tabular}{ll}
 \hline 
&GA solution\\ \hline 
&1575\\ \hline 
&1593\\ \hline 
&1609\\ \hline 
&1612\\ \hline 
&1606\\ \hline 
&1612\\ \hline 
&1618\\ \hline 
&1619\\ \hline 
&1624\\ \hline 
&1631\\ \hline \hline
Average&1609.9\\ \hline 
\end{tabular}
\end{align}

The average is close to the result of 1602 for instance E that we have used in
table \ref{tab:seeds}, but there does seem to be a lot to be gained if
we select the best solution.

To see how big an impact the number of generations has as compared to
starting with the right seeds, we ran the same experiment with 1000
generations. The results are shown in table \ref{tab:ga_repeat_1000} 

\begin{align}
\label{tab:ga_repeat_1000}
\begin{tabular}{ll}
\hline
&GA solution\\ \hline 
&1601\\ \hline 
&1601\\ \hline 
&1612\\ \hline 
&1624\\ \hline 
&1637\\ \hline 
&1640\\ \hline 
&1642\\ \hline 
&1652\\ \hline 
&1660\\ \hline 
&1764\\ \hline \hline
Average&1643.3\\ \hline 
\end{tabular}
\end{align}
As can be seen the results are significantly better than the results
with 100 generations. Even discounting the anomalously high solution
of 1764, the solutions are on average higher. This tells us that a
high number of generations is more important than restarting the
algorithm (or running it in parallel). A combined strategy of many
restarts and many generations seem like a good idea and should provide
good results. We have not been running with a higher amount of
generations since our main goal was to prove the feasibility of our
strategy and not necessarily achieve an exact measure of how well it
performed.

\subsection{Summing up the knapsack problem results}
Generally we think that our results with the knapsack problem have
been good. We have improved on the fat solution, sometimes a lot,
sometimes not so much. We have seen that our concept of seeding works
well, and generates better solutions than if we didn't use it.
We have also seen that we could improve on our results by running the
algorithm several times 



%\subsection{Mutation rate}
%\label{sec:experiments_mutation}
%
%\subsection{Running Time}
%\label{sec:running_time_knapsack}




\section{Capacitated lot-sizing problem experimentation}
\label{sec:experiments_lsp}
In order to test our methods for finding robust solutions to the
capacitated lot-sizing problem with uncertainty, we created a set
of instances with uncertainty. This set of uncertain instances is
based on a set of deterministic instances by Miller
\cite{tablesandmiller}. 

The instances by Miller have either $6$, $12$ or $24$
items, and either $15$ or $30$ time periods. They are single resource
instances of the capacitated lot-sizing problem, with setup times. 

We have tried solving all of these deterministic instances using a
branch-and-cut algorithm developed by Simon Spoorendonk \cite{spooren}
for his Ph.D. thesis, generally succeeding for instances with $6$
items and $15$ time periods. We have therefore derived a set of problem
instances with uncertainty that have $6$ items and $15$ timeperiods
based on the instance by Miller.

The instances have either uncertain \emph{capacity} (such instances
use the word ``cap'') or uncertain \emph{setup times} (instances use
the word ``setup'') or both (instances use the word
``cap-setup''). The instance marked ``deterministic'' is the original
instance by Miller. Please consult table \ref{tab:instances} for an
overview of these instances.

We have modelled uncertainty using \emph{normal distributions}, where
the mean value for all uncertain data equals the constant value in the
deterministic instance. We operate with two levels of standard
deviation. Instances with a ``big'' standard deviation a standard
deviation on uncertain data that is 25\% of the mean value. These instances have the
letter ``b'' appended. Instances with a ``small'' standard deviation
for all uncertain data, have a standard deviation that is 5\% of the
mean value. These instances have the letter ``s'' appended.

\begin{table}[!hbp]
\begin{tabular}{lll}
\hline
Instance 	& Uncertain capacity  	& Uncertain setup times \\
\hline
\hline
deterministic   &                       &     \\
cap-b           & yes                   &     \\
cap-s           & yes                   &     \\
setup-b         &                       & yes \\
setup-s         &                       & yes \\
cap-setup-b     & yes                   & yes \\
cap-setup-s     & yes                   & yes \\
\hline
\end{tabular}
\caption{Overview of instances with uncertainty. All instances have $6$ items and $15$ timeperiods.}
\label{tab:instances}
\end{table}

The goal of the experimentation was to find robust solutions for all
of the instances listed in table \ref{tab:instances}, and measure the
increase in cost compared to the deterministic solution.

\subsection{Results for genetic algorithm}
\label{sec:exp_ga_lsp}

We attempted an implementation of a genetic algorithm for the
capacitated lot-sizing problem with uncertainty, and ran it on the
problem instances. Due perhaps to the complexity of the problem we did
not succeed in getting the algorithm to converge to reasonably priced
robust solutions, i.e. solutions that produce items instead of simply
adding items to initial storage (the problem with such solutions being
that they are very expensive).

We implemented a two stage genetic algorithm, where the objective of
the first stage was to improve only the robustness of the population
without considering cost. When a threshold for robust individuals in
population was reached, we switched to stage two. In the second stage
of the algorithm, cost was optimized, while attempting to maintain the
robustness of the solutions found in stage one.

We attempted to seed the genetic algorithm with solutions to randomly
sampled scenarios, but this did not give us the edge that was needed
to make the algorithm convergence towards robust low-cost solutions.

\subsection{Results for scenario optimization}
\label{sec:exp_ga_lsp}

A different approach that we tried for the capacitated lot-sizing
problem with uncertainty, was scenario optimization
\cite{rockafellar}. In this approach we generated a set of
``pessimistic'' deterministic scenarios for each instance and then solved these to
optimility using the algorithm provided by Simon Spoorendonk. The
scenarios where constructed in such a way, that it was reasonable to
expect the optimal solutions to be robust in the instance.

We then used high precision simulation in order to quantify the
robustness of these solution in each of the instances listed in table
\ref{tab:instances}. We tried all solutions in all instances, i.e.
also solutions that had been made specifically for other instances.

Using a varying degree of pessimism when generating the scenarios, and
then solving these scenarios to optimality, we obtained a fairly diverse set
of solutions. 

Naturally all the solutions where $100\%$ robust when tested against
the deterministic instance \emph{deterministic}. 

Looking only at the instances with uncertainty, the deterministic
solution had the highest robustness in the instance with small
variation on the setup time, which is not surprising.

The deterministic solution had the lowest robustness in the uncertain
instance with big variation on the capacity. This is also not
surprising since uncertain capacity potentially limits the production
more than uncertain setup times. This is because uncertain capacities
have higher mean values than the setup times do. This means that
capacities have a bigger standard deviation in the stochastic model
and therefore have a bigger absolute span on realized values.

For every uncertain instance, at least one of the solutions obtained
by solving the pessimistic scenarios, was robust and had a reasonably
good objective value.

The solution to the original instance by Miller is refered
to as \emph{opt-deterministic}, while the optimal solutions to the pessimistic
scenarios use a naming scheme that links to the way the pessimistic
scenario was generated.

\subsubsection{Generating pessimistic scenarios}
When generating the pessimistic scenarios, we use values of $\chi \geq
\rho$, $\chi = \rho$ and $\chi < \rho$ (see section \ref{sec:methods_scenario_optimization} for details on
generating a pessimistic scenario).

We name the scenarios according to the value of $\chi$ and according to which data
is considered uncertain. In table \ref{tab:pessimistic_scenarios} we
give an overview of pessimistic scenarios for just one uncertain
instance (instance \emph{cap-b}). Pessimistic scenarios for
other instances use the same naming scheme.

\begin{table}[!hbp]
\begin{tabular}{lll}
\hline
Scenario                 & Optimal solution & Value of $\chi$          \\
\hline
\hline
scenario-nroot-rho-cap-b & nroot-rho-cap-b  & $\chi = \sqrt[n]{\rho}$ \\
scenario-fat-cap-b       & fat-cap-b        & $\chi = \rho $           \\
scenario-dim90-cap-b     & dim90-cap-b      & $\chi = \rho \cdot 0.9$  \\
scenario-dim80-cap-b     & dim80-cap-b      & $\chi = \rho \cdot 0.8$  \\
scenario-dim70-cap-b     & dim70-cap-b      & $\chi = \rho \cdot 0.7$  \\
\end{tabular}
\caption{Naming scheme for pessimistic scenarios derived from instance
\emph{cap-b}. The same naming scheme applies to pessimistic scenarios
when other instances are considered. The value $n$ is the number of
time periods in the lot-sizing problem, which is $15$ for all instances.}
\label{tab:pessimistic_scenarios}
\end{table}

It should be noted that we were not able to compute optimal solutions
to all scenarios, as the running time of the branch-and-cut algorithm
used is highly irregular, sometimes terminating within seconds and
other times within hours.

\subsubsection{Testing robustness and comparing objective value}

In table \ref{tab:deterministicsolution_robustness} we have summarized
the robustness of the deterministic solution in each of the uncertain
instances, with the understanding that a partial robustness of $0.95$
is needed in order for at solution to be considered robust. As can be
seen the deterministic solution is not robust in any of the uncertain
instances, which is what we expected.

Using simulation we have tested the robustness of all solutions in all
instances. For each instance we have measured the cost of the cheapest
robust solution, and compared it to the cost of the deterministic
solution.

\begin{table}[!hbp]
\begin{tabular}{lll}
\hline
Instance       & Robust & Bounds on partial robustness \\
\hline
\hline
deterministic  & Yes    &  [1.0, 1.0] \\
setup-s        & No     &  [0.62, 0.63] \\
setup-b        & No     &  [0.37, 0.39] \\
cap-s          & No     &  [0.22, 0.24] \\
cap-setup-s    & No     &  [0.22,  0.23] \\
cap-setup-b    & No     &  [0.009, 0.011] \\
cap-b          & No     &  [0.008, 0.011] \\
\hline
\end{tabular}
\caption{Robustness of the optimal deterministic solution. Not
surprisingly it is $100\%$ robust in the deterministic instance, and
not robust in any of the instances with uncertainty.}
\label{tab:deterministicsolution_robustness}
\end{table}

We have summarized the most robust solution found to each of the
instances, with data shown for partial robustness and cost compared to
the cost of the optimal deterministic solution.

\begin{table}[!hbp]
\begin{tabular}{lllllll}
\hline
Instance      & Best solution     & Robust  & Robustness bounds & $cost_0$ & $cost_{det}$ & $\frac{cost_0}{cost_{det}}$ \\
\hline
\hline
deterministic & opt-deterministic & Yes     & [1.0, 1.0]        & 8310429   & 50429   & $1.0$ \\
cap-b         & nroot-cap-b       & Yes     & [0.95, 0.96]      & 51890237  & 120237  & $6.2$ \\
cap-s         & nroot-cap-s       & Yes     & [0.97, 0.98]      & 10205879  & 55879   & $1.2$ \\ 
setup-s       & fat-setup-s 	  & Yes     & [0.99, 1.0]       & 8310810   & 50810   & $1.00005$  \\
setup-b       & fat-setup-b 	  & Yes     & [0.98, 0.99]      & 8310810   & 50810   & $1.00005$  \\
cap-setup-s   & dim80-cap-b 	  & Yes     & [0.98, 0.99]      & 12753835  & 53835   & $1.5$  \\
cap-setup-b   & nroot-cap-s 	  & Yes     & [0.96, 0.97] 	& 10205879  & 55879   & $1.2$  \\
\hline
\end{tabular}
\caption{Lowest cost robust solutions found for each instance of the
capacitated lot-sizing problem with uncertainty. If no robust solution
where found for an instance exists, the solution with the highest
partial robustness is shown. $cost_0$ is the cost of the solution
including cost of initial stock (which is $10000$ per item). $cost_1$
is the cost of the solution excluding the cost of initial stock.
$cost_det$ is the cost of the deterministic solution.}
\label{tab:instance-results}
\end{table}

When comparing tables \ref{tab:instance-results} and
\ref{tab:initial_stock}, it can be seen that the increased cost of a
robust solution is strongly correlated with the additional initial
stock used by the solution. While \emph{fat-setup-s} and
\emph{fat-setup-b} are both solutions which are much more robust than
the deterministic solution in instances \emph{setup-s} and
\emph{setup-b} correspondingly, they are not more expensive. This is
because the increased robustness is gained from a different production
schedule, rather than adding to initial stock. This is possible
because there is slack capacity that can be used in all time
periods. The very small increase in cost in these two solutions is
caused by shifting production slightly back in time, in order to meet
the demand. This incurs a small holding cost for storing items
produced a little earlier.

\begin{table}[!hbp]
\begin{tabular}{ll}
\hline
Best solution      & Initial stock (normalized) \\
\hline
\hline
opt-deterministic  & 1.0  \\
nroot-cap-b        & 6.3 \\
nroot-cap-s        & 1.2 \\ 
fat-setup-s 	   & 1.0  \\
fat-setup-b 	   & 1.0  \\
dim80-cap-b 	   & 1.5 \\
nroot-cap-s 	   & 1.2 \\
\hline
\end{tabular}
\caption{Initial stock used by best solutions.}
\label{tab:initial_stock}
\end{table}


