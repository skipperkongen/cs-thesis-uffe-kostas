\chapter{Solving problems under uncertainty}
\label{sec:solving_problems_under_uncertainty}

Having described the partially robust optimization problems in broad terms and
defined our use of terminology, we now turn to how uncertain problems
may be solved using conventional methods as well as how we plan to
solve them. We have identified three approaches to solving problems
under uncertainty.

\section{Scenario optimization}
\label{sec:scenario_optimization}

In chapter \ref{sec:scenarios} we introduced the concept of a scenario
$s \in \mathcal{S}_I$ given an uncertain problem $I$. In practice, a
common approach to finding robust solutions to uncertain problems is
to rely on scenario analysis \cite{rockafellar}. 

As described in chapter \ref{sec:scenarios}, a scenario is a
deterministic subproblem of the uncertain problem and can be modelled
as a regular linear program or integer program. A solution that holds
for a particular scenario can be found using a conventional,
deterministic solution method, and the robustness of this solution can
then be estimated using e.g. simulation.

In scenario optimization the uncertainty about parameters or
components of the system is modeled by a small number of subproblems
(scenarios). The idea is that by studying the different scenarios and
their optimal solutions, one may be able to discover similarities and
trends and eventually come up with a ``well hedged'' solution to the
uncertain problem. This can be expected to perform rather
well under all possible scenarios, relative to some weighting of scenarios
\cite{rockafellar}.

We describe our implementation of this approach in chapter
\ref{sec:methods_scenario_optimization}, where we describe how we
generate and solve a small set of representative scenario subproblems,
given an instance of an uncertain problem, and then proceed to
estimate the partial robustness of these solutions using simulation.

\subsection{Fat solutions}
\label{sec:fat_solutions}
A solution that can be expected to hold for all scenarios, is the
solution to the ``worst case'' scenario. We refer to such a solution
as the \emph{fat solution}.

We define the ``worst case'' scenario of a problem as the scenario
having the ``worst'' outcome for each of the stochastic elements.
The worst outcome of a stochastic element is the outcome that
constrains our solution space the most. Since this definition is more
suited for complete robustness rather than partial robustness, we use
a more relaxed definition.

\begin{definition}
The worst outcome of a stochastic element is the outcome that is worse
than $\rho$ of all outcomes of that stocastic element.
\end{definition}
Perhaps this can best be understood by an example
\begin{example}
We have a problem where wish to maximize the amount of items we can
bring with us, but we can take a maximum of 10 kg of items with us.
If an item has stochastic weight, then a higher weight will be worse
than a low one. Assume we have an item with equal distribution between
1 and 1000 grammes, then for $\rho=0.95$, 951 grammes would be the
worst outcome since it is worse than $1000 \rho = 950$ of the outcomes. 
\end{example}

The definition of worst may not be completely obvious for all
stochastic parameters but it is the case for the problems we are
working with.

We have now established that the value of each parameter considered
under uncertainty can be changed in a direction which can only make
the solution worse. With this we feel that it is well defined what we
mean by ``worst case'' scenario.

Assuming that uncertain parameters are normally distributed, we can
ensure that a solution is partially robust for some $\rho < 1$, by
finding the optimal solution to a worst-case scenario generated in the
following way. For all uncertain data $\pi$ in $(A,b)$, let $\pi$
assume a constant value equal to a threshold $v$ such that $v$ would
statistically dominate $\rho$ of all samples that were drawn from
the normal distribution associated with $\pi$. (see figure
\ref{fig:std_deviation} for various thresholds). This is of course not
possible for $\rho = 1$ as no absolute threshold exists that will
dominate all samples that can be drawn from a normal distribution.

The threshold we are looking for depends on $\rho$ and whether $v$
should be greater than $\rho$ of the samples drawn the distribution,
or less than $\rho$ of the samples, which again depends on whether
smaller or larger $\pi$ are considered bad (see example
\ref{ex:worst_case_example}). If smaller values of an uncertain
parameter $\pi$ have a negative impact on the objective value of
$x^*$, e.g. the capacity in an instance of the knapsack problem with
uncertain capacity, we would for the deterministic worst-case
scenario, choose a constant setting of the capacity such that this
value is statistically \emph{smaller} than $\rho$ of all outcomes of
the stochastic capacity.

For normal distributions this is possible as long as the uncertain
parameter $\rho < 1$.

\figur{In the normal distribution shown, the threshold at $\mu$ is
statistically greater than $50 \%$ of values that are drawn from the
distribution, while $\mu + 1 \sigma$ is $50 \% + 34.1 \% = 84.1 \%$
greater than such values, and $\mu + 2 \sigma$ is statistically greater than $84.1
\% + 13.6 \% = 97.7 \%$ of values sampled from the
distribution}{fig:std_deviation}{figurer/Standard_deviation_diagram2.eps}

By solving the problem of this worst-case scenario we get a \emph{fat solution}
that is ensured to have a partial robustness of at least $\rho$. The
fat solution will likely be partially robust also for $\rho$. The
downside is that this solution is very expensive.

When having unbounded uncertainty it is infeasible to require complete
robustness. In these cases
the worst case scenario would be impossible to formulate, as it would
be infinitely bad. We will be using discretized normal
distributions to model data, which are discrete, but infinite, and we
look for solutions that are less than 100$\%$ robust. This means that
we can formulate a worst-case scenario, because we can calculate the
value $v$ for each normal distribution that will dominate $\rho = 0.95$ of
samples from that normal distribution.

With these two alterations in mind we now redefine the term \emph{fat
solution} as we will be using it in our thesis.

\begin{definition} A fat solution is an optimal solution $x$
to a deterministic subproblem, the worst case scenario, $s \in
\mathcal{S}_I$ which has been derived from the uncertain problem $I$
in the following way: The deterministic problem is generated by
realizing the values of the stochastic parameters $(A,b) \in
\mathcal{U}$ of the uncertain problem such that the new deterministic
value is the \emph{best} value that is still \emph{worse} than $\rho$
of the possible outcomes of the stochastic parameter. 
\end{definition}
The terms worse and best are naturally dependant on the problem being
treated. If for instance we are trying to maximize profit and profit
is a stochastic parameter, then the worst profit is defined as: The
highest profit that is still lower than other sampled profits in
$\rho$ of the cases. This is the profit that we will choose for the
scenario that generates the fat solution. 

We will now present an example to help understand our definition of
the fat solution. This example will also demonstrate how the fat
solution is sometimes too expensive when looking for partially robust
solutions.

\begin{example}
\label{ex:worst_case_example}
Our example is based on the knapsack problem which will be described
in detail in chapter \ref{sec:knapsack_problem} but since this example
is very small it should be possible to make do without the formal
introduction of the problem. The example is the same as the one we
briefly sketched in chapter \ref{sec:introduction}.

We have a ship of stochastic size. The ship can carry between 171 and
270 containers. Each of the integer values in the range $[171,270]$
have the same probability, i.e. $0.01$. We have two customers who want
their orders shipped. The size of the customers' orders is also
stochastic and lies in the range $[1,100]$, also with a $0.01$
probability for each outcome. Our objective is now to bring orders from
as many customers as possible while still being $0.95$ certain that
the ship will be able to carry those orders, in other words, we want a
robustness $\rho = 0.95$.

The fat approach to this problem will be to assume that the ship has a
capacity of 176 since there exist 95 outcomes (of 100) that are as good or
better. The fat approach will also assume that each of the
customers need to ship 95 containers, again this is because there
exist 95 outcomes (of 100) that are as good or better. This is
illustrated in figure \ref{fig:fat_solution}.

\figur{Fat approach to the shipping problem. The green areas are the
  outcomes that are worse than the one selected. As can be seen the
  green areas represents only 5$\%$ of the possible outcomes for each
  stochastic element.}{fig:fat_solution}{figurer/fat_solution.eps}

Now the fat solution will be to bring only the order from one customer
since bringing the orders from both customers would result in a total
of 190 expected containers while we only have expected capacity for 176.

So the fat solution solves our problem, but is not an optimal
situation. We will now argue that we can indeed bring both orders and
still have a solution that is $0.95$ robust.

If we create a matrix of the possible combinations of containers from
the two customers with the total number of containers delivered by the
customers in each entry of the matrix it would look something like this:

\begin{center}
\begin{tabular}{l||l|l|lll|l|l|}
\textbf{Containers}   & 1 & 2 & \multicolumn{3}{c|}{$\cdots$} & 99 & 100 \\
\hline
\hline
1  & 2 & 3 & \multicolumn{3}{c|}{$\cdots$} & 100& 101 \\
\hline
2  & 3 & 4 & \multicolumn{3}{c|}{$\cdots$} & 101& 102 \\
\hline
$\vdots$ &$\vdots$&$\vdots$ &\multicolumn{3}{c|}{$\ddots$}&$\vdots$& $\vdots$\\
\hline
99  & 100 & 101 & \multicolumn{3}{c|}{$\cdots$} & 198&199\\
\hline
100 & 101 & 102 & \multicolumn{3}{c|}{$\cdots$} & 199&200\\
\hline
\end{tabular}
\end{center}

There are $10,000$ entries in the matrix ($100 \cdot 100$), giving
each entry a probability of $0.0001$. The important thing is to
look at the probability of each sum of containers. The probability of
$n$ containers being delivered is:
\[
P(C = n) = 100-(\vert 101-n \vert)\cdot 0.0001,\ n \in [2,200]
\]
i.e there is a $0.0001$ probability of 200 containers being delivered
from the two customers. The sum of containers are a multinomial
distribution, \cite{sandsynlighed}, and the main picture of the
distribution can be seen in figure \ref{fig:multinomial}. 

\figur{The distribution of the sum of containers being delivered by
  the two customers.}{fig:multinomial}{figurer/multinomial.eps}
With this information we can calculate the
probability of a certain amount of containers or more being delivered
from the customers. We now find the maximum amount of containers, $k$,
that satisfy 
\[
\sum_{i=k}^{200} P(C = i) \leq 0.05
\]
This happens to be 169 containers, thus if we plan for 168 containers
being delivered there will be less than $0.05$ probablity that we get more
containers from the customers. This way we can easily accept the
orders from both customers since we always have capacity for at least
168 containers.

The new situtaion is illustrated in figure
\ref{fig:fat_solution_improved}.

\figur{Improving on the fat solution by considering the sum of
  containers delivered by the customers. The dark green areas
  represent the improvement on assumed customer
  deliveries.}{fig:fat_solution_improved}{figurer/fat_solution_improved.eps}
\end{example}

One problem with the fat solution is that it requires that the
stochastic parameters are stochastically independent. For the
lot-sizing problem this is not necessarily the case and so the fat
solution is not necessarily robust. But for the knapsack problem it
works fine.

\section{Stochastic Programming}
\label{sec:stochastic_programming}

Another approach to solving problems with stochastic elements is to
use stochastic programming. In stochastic programming it is assumed
that there is some regularity to the stochastic element, which
corresponds nicely to our problems, where we assume an underlying
stochastic model based on primarily normal distributions. Stochastic
programming has several different approaches to solving the stochastic
problems. The most commonly mentioned is the two-stage linear programs.

\subsection{Two-stage linear programs}
In two-stage linear programming, we first make a decision based on
the information at hand, that is the constant, the variables from
earlier periods and the distribution of the stochastic
variables. After this decision has been made we await the outcome of
the stochastic variables. Once the outcome has been determined we are
then allowed to make a \emph{recourse} action which will allow us to
remedy whatever shortcomings our original decision had in the face of
the actual situation we are in now.

This means that a two-stage linear program has both an optimal
solution to the first stage problem as well as a number of recourse
actions for the second stage.

\subsection{Adding recourse to the objective function}
Another approach is to add the penalty incurred by the different
recourse actions into the objective function. The problem is then
converted to a linear program by inserting all possible realizations
of the stochastic values. This can naturally only be done for bounded
and finite distributions. The problem can then be solved as a linear
program. 

\subsection{Summing up on stochastic programming}

Stochastic programming may sound like just the thing for us, but the
problem is that small changes to the model will result in drastic
changes in how we solve the problem, and we are looking for a simpler
more generally applicable method.

In addition to this, stochastic programming works best with discrete
distributions and we would like to be able to work with continous
distributions.  

Since our aim is to be able to change both the number
of stochastic variables as well as how they are distributed,
stochastic programming does not seem like a feasible method for us.

%\section{Chance Constrained Programming}
%\label{sec:chance_constrained_programming}
%Chance constrained programming is a solution method that is very close
%to SP but differs in a few key aspects.
% Har droppet det da vi ikke kan skrive noget fornuftigt om det.

\section{Local search}
As the problem we want to solve has a non-convex solution space, we
need a solution method that does not require convexity. Local search
heuristics do not generally require convexity of the solution space.
Examples of local search algorithms are \emph{genetic algorithms},
\emph{simulated annealing}, \emph{hill climbing}, and \emph{taboo search}.

Tsutsui and Ghosh applied a genetic algorithm to robust optimization,
but did not have an absolute requirement regarding the robustness of
the solutions found, focusing on avoiding ``brittle'' solutions at
sharp peaks of the solutions space \cite{tsutsui}.

